
% taken from http://physics.clarku.edu/sip/tutorials/TeX/intro.html

\documentclass[12pt]{article}

\usepackage{amsmath}    % need for subequations
\usepackage{graphicx}   % need for figures
\usepackage{verbatim}   % useful for program listings
\usepackage{color}      % use if color is used in text
\usepackage{hyperref}   % use for hypertext links, including those to external documents and URLs

% don't need the following. simply use defaults
\setlength{\baselineskip}{16.0pt}    % 16 pt usual spacing between lines

\setlength{\parskip}{3pt plus 2pt}
\setlength{\parindent}{20pt}
\setlength{\oddsidemargin}{0.5cm}
\setlength{\evensidemargin}{0.5cm}
\setlength{\marginparsep}{0.75cm}
\setlength{\marginparwidth}{2.5cm}
\setlength{\marginparpush}{1.0cm}
\setlength{\textwidth}{150mm}

\begin{comment}
\pagestyle{empty} % use if page numbers not wanted
\end{comment}

% above is the preamble

\begin{document}

\begin{center}
{\large A Comprehensive Comparison of C++ and R} \\ % \\ = new line
\copyright 2020 by Cang and Navarez \\
November 2020
\end{center}

\section{R}
\subsection{Purpose and Motivations}
The first versions of R were ``clones'' of the S-PLUS language but were free.
\subsection{History (Authors, Revisions, Adoption)}
\subsubsection{The S programming Language}
S is a language that was developed as an internal statistical analysis environment by John Chambers at Bell Labs
\subsection{Language Features}
\subsection{Paradigm(s)}
\subsection{Language Evaluation Criteria}

\section{C++}
\subsection{Purpose and Motivations}
\subsection{History (Authors, Revisions, Adoption)}
\subsection{Language Features}
\subsection{Paradigm(s)}
\subsection{Language Evaluation Criteria}

\end{document}

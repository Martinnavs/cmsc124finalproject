
% taken from http://physics.clarku.edu/sip/tutorials/TeX/intro.html

\documentclass[12pt]{article}

\usepackage{amsmath}    % need for subequations
\usepackage{graphicx}   % need for figures
\usepackage{verbatim}   % useful for program listings
\usepackage{color}      % use if color is used in text
\usepackage{hyperref}   % use for hypertext links, including those to external documents and URLs

% don't need the following. simply use defaults
\setlength{\baselineskip}{16.0pt}    % 16 pt usual spacing between lines

\setlength{\parskip}{3pt plus 2pt}
\setlength{\parindent}{20pt}
\setlength{\oddsidemargin}{0.5cm}
\setlength{\evensidemargin}{0.5cm}
\setlength{\marginparsep}{0.75cm}
\setlength{\marginparwidth}{2.5cm}
\setlength{\marginparpush}{1.0cm}
\setlength{\textwidth}{150mm}

\begin{comment}
\pagestyle{empty} % use if page numbers not wanted
\end{comment}

% above is the preamble

\begin{document}

\begin{center}
{\large A Comprehensive Comparison of C++ and R} \\ % \\ = new line
\copyright 2020 by Cang, K. and Navarez, A. \\
November 2020
\end{center}

\tableofcontents
\newpage

\section{R}
\subsection{Purpose and Motivations}
Fundamentally, R is a dialect of S. It was created to do away with the limitations of S, which is that it is only available commercially.
\subsection{History (Authors, Revisions, Adoption)}
\subsubsection{S, the precursor of R}
S is a language created by John Chambers and others at Bell Labs on 1976. The purpose of the language was to be an internal statistical analysis environment. The first version was implemented by using FORTRAN libraries. This was later changed to C at S version 3 (1988), which resembles the current systems of R.

On 1988, Bell labs provided StatSci (which was later named Insigthful Corp.) exclusive license to develop and sell the language. Insightful formally gained ownership of S when it purchased the language from Lucent for \$ 2,000,000, and created the language as a product called S-PLUS. It was names so as there were additions to the features, most of which are GUIs.

\subsubsection{R}
R was created on 1991 by Ross Ihaka and Robert Gentleman of the University of Auckland as an implementation of the S language. It was presented to the 1996 issue of the \textit{Journal of Computational and Graphical Statistics} as a ``language for data analysis and graphics''. It was made free source when Martin Machler convinced Ross and Robert to include R under the GNU General Public License.

The first R developer groups were in 1996 with the establishment of R-help and R-devel mailing lists. The R Core Group was formed in 1997 which included associates which come from S and S-PLUS. The group is in charge of controlling the source code for the language and checking changes made to the R source tree.

R version 1.0.0 was publicly released in 2000. As of the moment of writing this paper, the R is in version 4.0.3.
\subsection{Language Features}
R as a language follows the philosophy of S, which was primarily developed for data analysis rather than programming. Both S and R have interactive environment that could easily service both data analysis (skewed to command-line commands) and longer programs (following traditional programming). R has the following features:
\begin{itemize}
\item Runs in almost every standard computing platform  and operating systems
\item Open-source
\item An effective data handling and storage facility
\item A suite of operators for calculations on array, in particular matrices
\item A large, coherent, integrated collection of intermediate tools for data analysis
\item Graphical facilities for data analysis and display either on-screen or on hardcopy u
\item Well-developed, simple, and effective programming language which includes conditionals, loops, user-defined recursive functions and input and output facilities.
\item Can be linked to C, C++, and Fortran for computationally-intensive tasks
\item A broad selection of packages available in the CRAN sites which cater a wide variety of modern statistics
\item An own LaTex-like documentation format to supply comprehensive documentation
\item Active community support
\end{itemize}

\subsection{Paradigm(s)}
\subsection{Language Evaluation Criteria}

\section{C++}
\subsection{Purpose and Motivations}
\subsection{History (Authors, Revisions, Adoption)}
\subsection{Language Features}
\subsection{Paradigm(s)}
\subsection{Language Evaluation Criteria}

\end{document}
